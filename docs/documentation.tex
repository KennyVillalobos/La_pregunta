\documentclass[article]{llncs}
%
\usepackage[utf8]{inputenc}
\usepackage[spanish]{babel}
\usepackage{graphicx}
% Used for displaying a sample figure. If possible, figure files should
% be included in EPS format.
%
% If you use the hyperref package, please uncomment the following line
% to display URLs in blue roman font according to Springer's eBook style:
% \renewcommand\UrlFont{\color{blue}\rmfamily}


\begin{document}
%
\title{Dise\~{n}o y An\'alisis de Algoritmos. Problema 3: La Pregunta}
%
%\titlerunning{Abbreviated paper title}
% If the paper title is too long for the running head, you can set
% an abbreviated paper title here
%
\author{Jes\'us Santos Capote y Kenny Villalobos Morales}
%
\institute{Facultad de Matemática y Computación, Universidad de La Habana, La Habana, Cuba }
%
\maketitle              % typeset the header of the contribution
%
\section{Definici\'on del Problema}

Se tiene una expresión booleana y se quiere conocer si existe una asignación a las variables de esta, tal que la expresión se
torne falsa.

\section{Primera Aproximaci\'on}

Dada la expresión brindada, sea esta $f$, si existe una asignación $s$ de las variables de $f$ tal que la expresión sea falsa,
entonces dicha asignación hace verdadera la expresión $\lnot f$, por lo que el problema de encontrar una asignación de variables que
hagan falsa $f$ sea soluciona hallando una asignación de variables que satifaga $\lnot f$. Para verificar la satisfacibilidad de
la fórmula $f$ es más sencillo trabajar sobre una fórmula en CNF, por lo cual hacemos la transformación de $f$ a $f'$, siendo $f'$
una fórmula en CNF equisatisfacible con $f$. Para esta transformación utilizamos el algoritmo Tseitin, que nos garantiza que dicha
transformación se desarrolla en tiempo polinomial, y con un crecimiento de la fórmula polinomial con respecto al tamaño de la
fórmula original.(Ver algoritmo de Tseitin en ...)

En este punto buscamos una asignación de variables tal que la fórmula $f'$ quede satisfecha, por tanto, procedemos a verificar todas
las posibles asignaciones de variables



\subsection{Idea del Algoritmo}





\subsection{Correctitud}


\subsection{Complejidad Temporal}



\section{Reducción del Problema}

Sea $A$ el problema que estamos tratando de resolver, y sea $B$$ el conocido problema NP-Completo 3-Sat.(Ver problema 3-Sat) Supongamos que tenemos un algoritmo $f$
que soluciona $A$ en tiempo polinomial. Para toda instancia $\beta$ de $B$ podemos transformar esta en una instancia $\alpha $ de $A$, de forma que la solución
de las instancias $\alpha $ de $A$ y $\beta $ de $B$ son la misma, es decir la respuesta de $\alpha $ es "si" si y solo si
la respuesta de $\beta $ es si. Este procedimiento de reducción polinomial es tan sencillo como tomar para cada instancia $\beta $ de 3-Sat
y convertirla a $\alpha $, siendo $\alpha  = \lnot $. Si el algoritmo $f$ retorna $True$, es porque existe una asignación de variables
para $\alpha $ tal que está es $False$, luego para esta misma asignación de variables $lnot \beta  = False$, por lo cual $\beta  = True$
\section{Algoritmo Walk-Sat}



\subsection{Idea del algoritmo}

\subsection{Criterios greedy}

\subsection{Eficacia del algoritmo}



\end{document}