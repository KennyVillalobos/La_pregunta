\documentclass[article]{llncs}
%
\usepackage[utf8]{inputenc}
\usepackage[spanish]{babel}
\usepackage{graphicx}
% Used for displaying a sample figure. If possible, figure files should
% be included in EPS format.
%
% If you use the hyperref package, please uncomment the following line
% to display URLs in blue roman font according to Springer's eBook style:
% \renewcommand\UrlFont{\color{blue}\rmfamily}


\begin{document}
%
\title{Dise\~{n}o y An\'alisis de Algoritmos. Problema 3: La Pregunta}
%
%\titlerunning{Abbreviated paper title}
% If the paper title is too long for the running head, you can set
% an abbreviated paper title here
%
\author{Jes\'us Santos Capote y Kenny Villalobos Morales}
%
\institute{Facultad de Matemática y Computación, Universidad de La Habana, La Habana, Cuba }
%
\maketitle              % typeset the header of the contribution
%
\section{Definici\'on del Problema}

Se tiene una expresión booleana y se quiere conocer si existe una asignación a las variables de esta, tal que la expresión se
torne falsa.

\section{Primera Aproximaci\'on}

Dada la expresión brindada, sea esta $f$, si existe una asignación $s$ de las variables de $f$ tal que la expresión sea falsa,
entonces dicha asignación hace verdadera la expresión $\lnot f$, por lo que el problema de encontrar una asignación de variables que
hagan falsa $f$ sea soluciona hallando una asignación de variables que satifaga $\lnot f$. Para verificar la satisfacibilidad de
la fórmula $f$ es más sencillo trabajar sobre una fórmula en CNF, por lo cual hacemos la transformación de $f$ a $f'$, siendo $f'$
una fórmula en CNF equisatisfacible con $f$. Para esta transformación utilizamos el algoritmo Tseitin, que nos garantiza que dicha
transformación se desarrolla en tiempo polinomial, y con un crecimiento de la fórmula polinomial con respecto al tamaño de la
fórmula original.(Ver algoritmo de Tseitin en ...)

En este punto buscamos una asignación de variables tal que la fórmula $f'$ quede satisfecha, por tanto, procedemos a verificar todas
las posibles asignaciones de variables



\subsection{Idea del Algoritmo}

La idea de esta aproximaci\'on para encontrar una distribuci\'on de valores de las variables, que satisfagan la CNF es 
mediante Backtrack, probar todas las posibles asignaciones de las variables y comprobar si alguna de ellas hace 
que la CNF sea satisfacible. Si no existe tal distribuci\'on el algoritmo lo notifica.

\subsection{Correctitud}
Este algoritmo es correcto pues se comprueban todas las posibles asignaciones de las variables. Por tanto, si existe 
una distribuci\'on de valores que hace que la CNF se satisfaga, el algoritmo la detecta. Tambi\'en detecta si no existe 
forma de satisfacer la CNF.

\subsection{Complejidad Temporal}
Sea $n$ la cantidad de variables. Sea $c$ la cantidad de cl\'ausulas de la CNF. La cantidad de distribuciones de valores booleanos 
de las variables de la CNF es $2^n$. Luego para 
cada distribuci\'on se comprueba si satisface la CNF. La comprobaci\'on de satisfacibilidad tiene complejidad 
$O(n*c)$ pues se recorren todas las variables de todas las cl\'ausulas asignando sus valores y evaluando las cl\'ausulas durante el recorrido y luego se 
verifica el valor de las evaluciones de todas las cl\'ausulas en $O(c)$ la cnf durante el recorrido. Por tanto el algoritmo tiene complejidad $O(n*c*2^n)$


\section{Reducción del Problema}

Sea $A$ el problema que estamos tratando de resolver, y sea $B$ el conocido problema NP-Completo 3-Sat.(Ver problema 3-Sat) Supongamos que tenemos un algoritmo $f$
que soluciona $A$ en tiempo polinomial. Para toda instancia $\beta$ de $B$ podemos transformar esta en una instancia $\alpha $ de $A$, de forma que la solución
de las instancias $\alpha $ de $A$ y $\beta $ de $B$ son la misma, es decir la respuesta de $\alpha $ es "si" si y solo si
la respuesta de $\beta $ es si. Este procedimiento de reducción polinomial es tan sencillo como tomar para cada instancia $\beta $ de 3-Sat
y convertirla a $\alpha $, siendo $\alpha  = \lnot $. Si el algoritmo $f$ retorna $True$, es porque existe una asignación de variables
para $\alpha $ tal que está es $False$, luego para esta misma asignación de variables $lnot \beta  = False$, por lo cual $\beta  = True$

\section{Algoritmo Walk-Sat}


\section{Utilización del algoritmo Walk-Sat}

Walk-Sat es un algoritmo estocástico de búsqueda local(focused search algorithm), en los algoritmos de búsqueda local o enfocada, se elige la nueva 
variable al azar entre las variables que aparecen en las cláusulas violadas, con una probabilidad $1-p$, mientras que con una probabilidad $p$
 se escoge basado en un criterio greedy. Es uno de los algoritmos más estudiados para solucionar problemas SAT sobre CNF.
Para llevar una instancia de nuestro problema a una instancia de 3-SAT y utilizar Walk-Sat en timempo polinomial tomamos la fórmula original 
$f$ y la negamos, siendo $f' = \lnot f$, y utilizamos la transformación de Tseitin sobre $f'$, obteniendo una fórmula $f''$ dado que esta
nos permite llevar una expresión booleana a 3-CNF en tiempo polinomial y con un crecimiento de la fórmula polinomial con
respecto al tamaño de la fórmula original. Luego resolver 3-SAT sobre la fórmula $f''$ implica encontrar que existe una asignación de
variables que satisface la fórmula $f'$, que significa que existe una asignación de variables para la cual $f$ toma valor $False$.

\subsection{Idea del algoritmo}

Comienza asignando un valor aleatorio a cada variable de la fórmula. Si la asignación satisface todas las cláusulas, 
el algoritmo termina devolviendo la asignación. De lo contrario, se invierte una variable seleccionada bajo un criterio de
una cláusula insatisfecha elegida aleatoriamente y se repite lo anterior hasta que todas las cláusulas estén satisfechas o se llegue a
un límite establecido de número de iteraciones, para el cual se da como aproximación a la solución la distribución hallada hasta ese punto 
y el número de cláusulas insatisfechas. El algoritmo puede utilizar diferentes criterios, puede seleccionar una variable de la cláusula de 
forma aleatoria (RandomWalkSAT), o bien puede seguir un criterio greedy de los implementados para seleccionar la variable.

\subsection{Criterios greedy}

\subsubsection{Minimizar la insatisfacción de cláusula que ya estaban satisfechas.}

Este criterio escoge la variable de la cláusula que aparezca menos veces como única variable que satisface una cláusula, dado que cambiar esta
haría que menos cláusulas se insatisfagan.

\subsubsection{Maximizar la cantidad de cláusulas que se satisfacen}

Este criterio escoge la variable de la cláusula que aparezca en más cláusulas insatisfechas, dado que cambiar esta satisfacería
dichas cláusulas.

\subsection{Eficacia del algoritmo}

Christos H. Papadimitriou en "On selecting a satisfying truth assignment", Papadimitriou demostró que RandomWalkSAT 
resuelve con alta probabilidad cualquier instancia satisfactible del conjunto aleatorio 2-SAT, en un número de 
pasos $T = O(N^2)$. Además, Uwe Schöning en "A probabilistic algorithm for k-sat based on limited local
search and restart" demostró que: sea $R$ el número de reinicios del algoritmo RandomWalkSAT, cada vez con una condición 
inicial dibujada uniformemente al azar. Sea $T = 3N$ el número de variables volteadas de cada reinicio. Si ninguna de 
las R ejecuciones sucesivas de RandomWalkSAT en la instancia F ha proporcionado una solución, entonces la probabilidad 
de que F sea satisfactible es menor que $e − (3/4)NR$. Por lo tanto, después de muchos reinicios $R > (4/3)N$, si hay soluciones, 
la probabilidad de que RandomWalkSAT no haya encontrado ninguna de ellas es muy baja. Otros estudios sobre algoritmo como 
"Solving satisfiability problems by fluctuations: The dynamics of stochastic local search algorithms"(Wolfgang Barthel, 
Alexander K. Hartmann, and Martin Weigt), referencian que el tiempo de ejecución depende en gran medida de la relación 
$\alpha = M/N$, con $M$ siendo el número de cláusulas y $N$ el número de variables, para valores pequeños de $\alpha $, en el  
algoritmo RandomWalkSAT, negar una variable en una cláusula insatisfecha raramente hace que otras cláusulas se vuelvan insatisfechas. 
Hasta un umbral crítico $\alpha_d \simeq  2.7$ se encuentra una solución en un tiempo mediano que crece linealmente con N, por encima de este 
punto los tiempos de ejecución crecen exponencialmente. Si elegimos una heurística mejor que tenga un número de equilibrio menor de cláusulas insatisfechas, 
al introducir una fracción $q > 0$ de pasos greedy, el umbral dinámico en sí cambia ligeramente y tiene un máximo en $q \simeq  0.85$. Para fórmulas de hasta 
$\alpha  \simeq  2.8$ se pueden resolver en tiempo lineal.


\section{Contenidos preliminares.}

\subsection{Problema de satisfacibilidad booleana. 3-SAT.}

El problema SAT es el problema de saber si, dada una expresión booleana con variables y sin cuantificadores, 
hay alguna asignación de valores para sus variables que hace que la expresión sea verdadera. El problema 3-SAT restringe 
la expresión de entrada a Forma Normal Conjuntiva(CNF) con 3 variables por cláusula (3-CNF).

\subsection{Algoritmo de Tseitin}

Dada una fórmula $\phi $ identificamos todas las subfórmulas $\phi_i$ que la componen (subfórmulas entre dos variables a lo sumo) e introducimos una nueva variable 
$y_i$ por cada una de estas. Luego reescribimos la fórmula original como la consecución de $and$ entre la la variable correspondiente
a la subfórmula mayor y las clásulas de la forma: $ y_i \Leftrightarrow \phi_i$. Luego, cada una de estas cláusulas debe ser llevada
a CNF, utilizando el algoritmo de transformación basado en el uso de la tabla verititativa, como cada cláusula tiene tres variables a lo sumo, su representación
en FNC dada por este método es un expresión en con menos de tres variables por cláusula. Luego para conseguir que cada clásula $C_i$ tenga exactamente
3 variables, se sigue el procedimiento:

\bullet Si $C_i$ tiene 3 literales distintos, simplemente incluimos $C_i$ como una cláusula de la fórmula original.

\bullet Si $C_i$ tiene 2 literales distintos, dígase $C_i = (l_1 \vee  l_2)$, entonces incluimos como cláusulas
$(l_1\vee l_2\vee p)\wedge (l_1\vee l_2\vee \lnot p)$

\bullet Si $C_i$ tiene soll un literal $l$, entonces incluimos las cláusulas 
$(l\vee p\vee q)\wedge (l\vee p\vee \lnot q) \wedge (l\vee \lnot p \vee q)\wedge (l\vee \lnot p\vee \lnot q)$

\end{document}